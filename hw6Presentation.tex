\documentclass{beamer}
\usepackage[utf8]{inputenc}
\usepackage{tikz}
\usepackage{listings}
\usepackage{color}
\usepackage{float} 
 
\definecolor{javared}{rgb}{0.6,0,0} % for strings
\definecolor{javagreen}{rgb}{0.25,0.5,0.35} % comments
\definecolor{javapurple}{rgb}{0.5,0,0.35} % keywords
\definecolor{javadocblue}{rgb}{0.25,0.35,0.75} % javadoc

\lstset{language=Java,
basicstyle=\small,
keywordstyle=\color{javapurple}\bfseries,
stringstyle=\color{javared},
commentstyle=\color{javagreen},
morecomment=[s][\color{javadocblue}]{/**}{*/},
numbers=left,
numberstyle=\small\color{black},
stepnumber=1,
numbersep=10pt,
tabsize=4,
showspaces=false,
showstringspaces=false}

\title{\includegraphics[scale=.35]{FacelookCALogo.pdf}\\
The Cellular Automata Framework for
the Next Generation}
\author{Benjamin Chung, Hank Zwally, Cory Williams}
\usetheme[height=7mm]{Rochester}
\begin{document}
\maketitle
\begin{frame}{Introduction}
\begin{itemize}
  \item FacelookCA is designed to be fast, easy, and expandable, while still providing
great features.
\item FacelookCA includes many components that you would otherwise have to
write
\item FacelookCA supports multi-threaded execution and other advanced features,
while still providing a simple easy-to-use interface.
\end{itemize}
\end{frame}
\begin{frame}[fragile]{Rules}
As a plugin developer, your main task is to write rules. FacelookCA was designed
with just this in mind. To implement a rule, you write just one function.
\begin{exampleblock}{Game of life in 12 lines}
\begin{lstlisting}
@Override
public int getState(Board board, int x, int y) {
    int neighbors = 
        Utils.getNeighborsWithState(board, 1, x, y);
    int state = board.getState(x, y);
    if (state == 0 && neighbors == 3) 
        return 1;
    if (state == 1 && 
        (neighbors == 2 || neighbors == 3))
        return 1;
    return 0;
}
\end{lstlisting}
\end{exampleblock}
\end{frame}
\begin{frame}{Runners}
A runner is how your rule meets the board. FacelookCA comes with 2 runners. Via
the runner system, FacelookCA offers spectacular speed with little effort on the part of the rule developer.
\begin{itemize}
  \item SimpleRunner - Runs the cellular automata is a single thread.
  \item MTBoardRunner - Multithreaded runner, same interface!
\end{itemize}
\end{frame}
\begin{frame}{Views}
FacelookCA provides a hugely expandable and easy to use system of views. A cell
renderer can render whatever it wants to a Swing panel, or very quickly to a
bitmap.

\begin{itemize}
  \item Swing - 2 FPS
  \item Bitmap - 15 FPS
\end{itemize}
\end{frame}
\begin{frame}{Loaders}
FacelookCA supports the run length encoded standard for cellular automata
storage. In addition, the framework lets you easily extend its built in file
parsing functionality through a simple and easy to grasp interface.
\end{frame}
\begin{frame}{Built in functionality}
FacelookCA has most of the functionality that you need to run a cellular
automata built in. It already includes a renderer for a two state automaton, and
a built-in importer for the MCell format for Game of Life like automata.
\end{frame}
\end{document}