\documentclass{report}
\usepackage[utf8]{inputenc}
\usepackage[pdftex]{graphicx}
\usepackage{tikz}
\usepackage{listings}
\usepackage{color}
 \usepackage{float}

\definecolor{javared}{rgb}{0.6,0,0} % for strings
\definecolor{javagreen}{rgb}{0.25,0.5,0.35} % comments
\definecolor{javapurple}{rgb}{0.5,0,0.35} % keywords
\definecolor{javadocblue}{rgb}{0.25,0.35,0.75} % javadoc


\lstset{language=Java,
basicstyle=\ttfamily,
keywordstyle=\color{javapurple}\bfseries,
stringstyle=\color{javared},
commentstyle=\color{javagreen},
morecomment=[s][\color{javadocblue}]{/**}{*/},
numbers=left,
numberstyle=\small\color{black},
stepnumber=1,
numbersep=10pt,
tabsize=4,
showspaces=false,
showstringspaces=false}


\title{FacelookCA - The next generation of Cellular Automata - Specification}
\author{Benjamin Chung}%more
\begin{document}
\maketitle
The FacelookCA framework is designed to be fast, flexible, and extendable. 
Towards these objectives, it has 3 features that go above and beyond a simple CA
framework.
\section{Speed}
FacelookCA uses two kinds of runners, a simple single-threaded runner, and a
multithreaded one. The determination as to which to use is left up to the Rule
author. In addition, future plans include the construction of a GPGPU runner to
allow certian types of rule to be executed on the GPU.
\section{Flexibillity}
FacelookCA is highly flexible in terms of the kind of rules that one can create.
As the board system it uses uses 32 bit integer values for storage, a
signifigant amount of state data can be stored in each cell. Langdon's ant and
wireworld CA systems are trivial and multithreadable thanks to this.
\section{Extendable}
FacelookCA is extremely extendable, as almost everything can be altered or added
by the client. Rules and Renderers are plugin modifyable, and
the framework can use them like ones created internally.
\end{document}